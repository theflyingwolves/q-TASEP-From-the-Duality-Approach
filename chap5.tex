\chapter{Conclusion}
In this thesis, the model q-TASEP is studied from the duality approach with main reference to \cite{duality2014} and \cite{asymptotics2013}. We've arrived at a nested contour integral formulation of the q-Laplace transform of the probability density for positions of a particular particle in q-TASEP, and we also performed some asymptotic analysis to show that the rescaled fluctuation of the particle around its macroscopic position approximation follows the GUE Tracy-Widom distribution.

Towards the end, it's remarked that it has been shown that the KPZ scaling theory conjecture is satisfied for q-TASEP. Details have been worked out in \cite{asymptotics2013} \textit{Conjecture 2.8} and \textit{Theorem 2.9}. Moreover, there are also other variations of the q-TASEP that we have studied, for example, the discrete time q-TASEP studied in \cite{discreteqtasep}.

In conclusion, main effort of the project was spent to understand the relevant materials. Along the way, almost every aspect of my knowledge in mathematics was applied somewhere. For example, real and complex analysis is one of the biggest components of the project and important to understand most of the techniques used for the analysis part. Residue calculus is also key for the contour manipulations. In addition, I've also learned a lot of new techniques in evaluating contour integrals and performing contour deformation, etc. It has brought my understanding in mathematics to a level deeper than ever before. 