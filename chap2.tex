\chapter{Duality and Nested Contour Integral}
In this chapter, we recall the general definition of duality between two Markov process defined in \cite{interacting-particle-system} and provide some remarks on the definition. We then prove the duality between q-TASEP and q-TAZRP and conclude by giving an ansatz solution for q-TASEP with step initial data as well as half stationary initial data in the form of nested contour integrals.

\section{Duality}
We begin by recalling the general definition of duality from $Definition 3.1$ of \cite{interacting-particle-system}.
\begin{definition}
Suppose $x(t)$ and $y(t)$ are two independent Markov processes with state space $X$ and $Y$ respectively, and let $H(x,y)$ be a bounded measurable function on $X\times Y$. Then the processes $x(t)$ and $y(t)$ are said to be \textbf{dual} to one another with respect to $H$ if $$\mathbb{E}^{x}[H(x(t),y)] = \mathbb{E}^{y}[H(x,y(t))]$$ for all $x \in X$ and $y \in Y$. Here $\mathbb{E}^x$ is the expectation taken with respect to the process $x(t)$ with initial data $x(0) = x$ and similarly for $\mathbb{E}^y$.
\end{definition}
Note that with duality, we can easily derive that the infinitesimal generator $\Omega^x$ and $\Omega^y$ of the processes $x(t)$ and $y(t)$, respectively, acting on the function $H(x,y)$ are equivalent, i.e., 
\begin{align*}
(\Omega^x H(\cdot,y))(x) &= \lim_{t \rightarrow 0} \frac{\mathbb{E}^{x}[H(x(t),y)] - H(x,y)}{t}\\
												&= \lim_{t \rightarrow 0} \frac{\mathbb{E}^{y}[H(x,y(t))] - H(x,y)}{t}\\
												&= (\Omega^x H(x,\cdot))(y)
\end{align*}
In fact, the reverse is also true under some weak conditions \cite{generator-duality}. That is, given that the infinitesimal generator of two Markov processes $x(t)$ and $y(t)$ acting on function $H(x,y)$ are equal, we can derive that the two processes are dual with respect to the function $H(x,y)$. 

\begin{theorem}
\label{thm-duality}
The q-TASEP, $\vec{x}(t)$ with state space $X^N$ and particle jump rate parameters $a_i>0$, and the q-TAZRP, $\vec{y}(t)$, with state space $Y^N$ and rate functions $g_i(k)=a_i(1-q^k)$ are dual with respect to $$H(\vec{x},\vec{y}) = \prod_{i=0}^{N} q^{(x_i + i)y_i}.$$
\end{theorem}

\begin{remark}
It is noted that in the definition, $x_0 = \infty$ as we have discussed before. Therefore, for any $\vec{y}$ with $y_0 > 0$, $H(\vec{x},\vec{y}) = 0$.
\end{remark}

\begin{lemma}
\label{lem:duality}
Assume q-TASEP $\vec{x}(t)$ with initial condition $\vec{x}$ and q-TAZRP $\vec{y}(t)$ with initial condition $\vec{y}$. Then $$(L^{q-TASEP} H(\cdot , \vec{y}))(\vec{x}) = (L^{q-TAZRP} H(\vec{x}, \cdot))(\vec{y})$$.
\end{lemma}
\begin{proof}
By definition,
\begin{align*}
(L^{q-TASEP} H(\cdot , \vec{y}))(\vec{x}) &= \sum_{i=1}^{N} a_i (1-q^{x_{i-1} - x_i - 1}) (H(\vec{x}_i^+,\vec{y}) - H(\vec{x},\vec{y}))\\
																				&= \sum_{i=1}^{N} a_i (1-q^{x_{i-1} - x_i - 1}) (\prod_{j=0,j \neq i}^{N} q^{(x_j+j)y_j} (q^{(x_i+1+i)y_i} - q^{(x_i + i)y_i}))\\
																				&= \sum_{i=1}^{N} a_i (1-q^{x_{i-1} - x_i - 1})(q^{y_i} - 1) H(\vec{x},\vec{y}) \numberthis \label{eqn:duality-tasep}
\end{align*}
Similarly, for q-TAZRP we also have the following
\begin{align*}
(L^{q-TAZRP} H(\vec{x}, \cdot))(\vec{y}) &= \sum_{i=1}^{N} a_i (1-q^{y_i}) (H(\vec{x},\vec{y}^{i,i-1}) - H(\vec{x},\vec{y}))\\
																				&= \sum_{i=1}^{N} a_i (1-q^{y_i})\prod_{j=0,j \neq i, i-1}^{N} q^{(x_j+j)y_j} \notag \\
																				&\hphantom{{} = 1} \times (q^{(x_i+i)(y_i-1)} q^{(x_{i-1}+i-1)(y_{i-1}+1)} - q^{(x_i + i)y_i} q^{(x_{i-1} + i-1)y_{i-1}}) \\
																				&= \sum_{i=1}^{N} a_i (1-q^{y_i}) (q^{x_{i-1} - x_i - 1} - 1) H(\vec{x},\vec{y}) \numberthis \label{eqn:duality-tazrp}
\end{align*}
Comparing Equation (\ref{eqn:duality-tasep}) and (\ref{eqn:duality-tazrp}), we conclude that $$(L^{q-TASEP} H(\cdot , \vec{y}))(\vec{x}) = (L^{q-TAZRP} H(\vec{x}, \cdot))(\vec{y}).$$
\end{proof}

Before we continue to the proof of Theorem \ref{thm-duality}, we define a system of ordinary differential equation:
\begin{definition}
\label{dfn:true-evolution-equation}
We say that $h(t;\vec{y}): \mathbb{R}_+ \times Y^N \rightarrow \mathbb{R}$ solves the \textbf{\emph{true evolution equation system}} with initial data $h_0(\vec{y})$ if:
\begin{enumerate}
\item[(1)] For all $\vec{y} \in Y^N$ and $t \in \mathbb{R}_+$, $$\frac{d}{dt} h(t;\vec{y}) = L^{q-TAZRP} h(t;\vec{y});$$
\item[(2)] For all $\vec{y} \in Y^N$ such that $y_0 > 0$, $h(t;\vec{y}) = 0$ for all $t \in \mathbb{R}_+$;
\item[(3)] For all $\vec{y} \in Y^N$, $h(0;\vec{y}) = h_0(\vec{y})$.
\end{enumerate}
\end{definition}

\begin{remark}
It is remarked that the existence and uniqueness of solutions to the \textbf{\emph{true evolution equation system}} is assured. This follows from elementary ODE theories since the system reduces to a finite number of differential equations. 
\end{remark}

\begin{proof}[Proof of Theorem \ref{thm-duality}:]
We prove the theorem by showing that both $\mathbb{E}^{\vec{x}}[H(\vec{x}(t),\vec{y})]$ and $\mathbb{E}^{\vec{y}}[H(\vec{x},\vec{y}(t))]$ is a solution to the \textbf{\emph{true evolution equation system}} as defined in Definition \ref{dfn:true-evolution-equation}. Then by uniqueness, duality follows. 

That $\mathbb{E}^{\vec{y}}[H(\vec{x},\vec{y}(t))]$ is a solution to the \textbf{\emph{true evolution equation system}} is clear from the fact that $$\frac{d}{dt}\mathbb{E}^{\vec{y}}[H(\vec{x},\vec{y}(t))] = \lim_{t \rightarrow 0} \frac{\mathbb{E}^{\vec{y}}[H(\vec{x},\vec{y}(t))] - H(\vec{x},\vec{y})}{t} = L^{q-TAZRP} \mathbb{E}^{\vec{y}}[H(\vec{x},\vec{y}(t))],$$ with initial and boundary conditions easily checked. 

It then suffices to show that $\mathbb{E}^{\vec{x}}[H(\vec{x}(t),\vec{y})]$ is a solution to the \textbf{\emph{true evolution equation system}}. Since the initial and boundary conditions can be easily checked, we only need to show that $\frac{d}{dt} \mathbb{E}^{\vec{x}}[H(\vec{x}(t),\vec{y})] = L^{q-TAZRP} \mathbb{E}^{\vec{x}}[H(\vec{x}(t),\vec{y})]$ as below:
\begin{align*}
\frac{d}{dt} \mathbb{E}^{\vec{x}}[H(\vec{x}(t),\vec{y})] &= L^{q-TASEP} \mathbb{E}^{\vec{x}}[H(\vec{x}(t),\vec{y})] \quad (\text{by definition})\\
																												 &= \mathbb{E}^{\vec{x}} [L^{q-TASEP} H(\vec{x}(t),\vec{y})] \quad (\text{by commutativity})\\
																												 &= \mathbb{E}^{\vec{x}} [L^{q-TAZRP} H(\vec{x}(t),\vec{y})] \quad (\text{by Lemma }\ref{lem:duality}) \\
																												 &= L^{q-TAZRP} \mathbb{E}^{\vec{x}} [H(\vec{x}(t),\vec{y})] \quad (\text{Since $L^{q-TAZRP}$ acts on $\vec{y}$})
\end{align*}
\end{proof}
Using the duality, in order to find $\mathbb{E}^{\vec{x}}[H(\vec{x}(t),\vec{y})]$, it suffices to find $\mathbb{E}^{\vec{y}}[H(\vec{x},\vec{y}(t))]$. Note that the difference between the two expression is that the former is evaluating an expectation in the context of a q-TASEP, while the latter is evaluating the expectation in the context of a q-TAZRP. In what follows, we are going to show how to evaluate the expectation $\mathbb{E}^{\vec{y}}[H(\vec{x},\vec{y}(t))]$ in q-TAZRP using the \emph{free evolution equation system}. With this, we then are able to get an explicit formula for $\mathbb{E}^{\vec{x}}[H(\vec{x}(t),\vec{y})]$.
\section{Evolution Equation Systems}
\label{evolution-equation-system}
In this section, we are going to define another evolution equation systems, namely, the \emph{free evolution equation system} and show that a solution to the \emph{free evolution equation} also solves the \emph{true evolution equation system} with certain initial data.

Before moving on to the definition of the \emph{free evolution equation system}, we introduce a notion $\nabla_i$ defined by $\nabla_i f(\vec{n}) = f(\vec{n}_i^-) - f(\vec{n})$ for $\vec{n} = (n_1, \dots, n_k)$, where $\vec{n}_i^- = (n_1,\dots,n_{i-1}, n_i - 1, n_{i+1}, \dots, n_k)$.

\begin{definition}
\label{def:free-evolution-equation}
Given $\vec{x} \in X^N$, the state space of a q-TASEP, we say that $u(t;\vec{n}):\mathbb{R}_+ \times (\mathbb{Z}_{\ge 0})^k \rightarrow \mathbb{R}$ solves the \textbf{\emph{free evolution equation system}} with $k-1$ boundary conditions if:
\begin{enumerate}
\item[(1)] For all $\vec{n} \in (\mathbb{Z}_{\ge 0})^k$ and $t \in \mathbb{R}$, $$\frac{d}{dt} u(t;\vec{n}) = (1-q) \sum_{i=1}^{k} a_{n_i} \nabla_i u(t;\vec{n});$$
\item[(2)] For all $\vec{n} \in (\mathbb{Z}_{\ge 0})^k$ such that for some $i \in \{1,...,k-1\}$, $n_i = n_{i+1}$, $$\nabla_i u(t;\vec{n}) = q \nabla_{i+1} u(t;\vec{n});$$
\item[(3)] For all $\vec{n} \in (\mathbb{Z}_{\ge 0})^k$ such that $n_k = 0$, $u(t;\vec{n}) = 0$ for all $t \in \mathbb{R}_+$;
\item[(4)] For all $\vec{n} \in W^k_{>0}$, $u(0,\vec{n}) = H(\vec{x},\vec{y}(\vec{n}))$. 
\end{enumerate}
\end{definition}

It's not clear up to now that a solution exists for the \emph{free evolution equation system}. However, we are going to show that if one exists, then it must also solve the \emph{true evolution equation system} with the corresponding initial data. This is stated as a proposition below:

\begin{proposition}
\label{free-to-true-evolution}
If $u(t;\vec{n}):\mathbb{R}_+ \times (\mathbb{Z}_{\ge 0})^k \rightarrow \mathbb{R}$ is a solution to the \emph{free evolution equation system} with $\vec{x} \in X^N$, then for all $\vec{y} \in Y^N$ such that $\sum_{i=1}^{N} y_i = k$, $$\mathbb{E}^{\vec{x}}[H(\vec{x}(t), \vec{y})] = u(t;\vec{n}(\vec{y})).$$
\end{proposition}

\begin{remark}
It should be noted that $u(t;\vec{n}(\vec{y}))$ is a restriction of the solution $u(t;\vec{n})$ with domain $\mathbb{R}_+ \times (\mathbb{Z}_{\ge 0})^k$ to the domain $\mathbb{R}_+ \times W_{>0}^k$. Therefore, we are not claiming the truthness of the inverse because we are not sure whether the inverse extension would be possible for any given $\vec{y} \in Y^N$.
\end{remark}

\begin{proof}
Recall that $\mathbb{E}^{\vec{x}}[H(\vec{x}(t), \vec{y})]$ is the solution to the \emph{true evolution equation system} with initial data $H(\vec{x}, \vec{y})$. In order to show that $\mathbb{E}^{\vec{x}}[H(\vec{x}(t), \vec{y})] = u(t;\vec{n}(\vec{y}))$, we only need to show that for any solution $u(t;\vec{n})$ of the \emph{free evolution equation system}, when restricted to the domain $\mathbb{R}_+ \times W_{>0}^k$, it also solves the \emph{true evolution equation system} with initial data $H(\vec{x},\vec{y}(\vec{n}))$, where $\sum_{i=1}^{N} y_i = k$. It then suffices to check $v(t;\vec{y}) = u(t; \vec{n}(\vec{y}))$ against the three conditions in Definition \ref{dfn:true-evolution-equation}. 
\begin{enumerate}
\item[(1)] For all $\vec{y} \in Y^N$ and $t \in \mathbb{R}^+$, we want to show that $$(1-q) \sum_{i=1}^{k} a_{n_i} \nabla_i u(t;\vec{n}) = \frac{d}{dt} u(t;\vec{n}(\vec{y}))= \frac{d}{dt} v(t;\vec{y}) = \sum_{i=1}^{N} a_i (1-q^{y_i}) (v(t;\vec{y}^{i,i-1}) - v(t;\vec{y})).$$ For this purpose, we claim that for each site $i$ with $y_i$ number of particles $n_{k_1} = n_{k_2} = \dots = n_{k_{y_i}} = i$, $$(1-q) \sum_{j=k_1}^{k_{y_i}} a_{i} \nabla_j u(t;\vec{n}) = a_i (1-q^{y_i}) (v(t;\vec{y}^{i,i-1}) - v(t;\vec{y})).$$ Then by summing up over $i$, all sites, we get the desired equality. 

For notational simplicity, we only prove the claim for $i = N$, and other sites work similarly. That is, we want to show that $$(1-q) \sum_{j=1}^{y_N} a_{N} \nabla_j u(t;\vec{n}) = a_N (1-q^{y_N}) \nabla_{y_N} u(t;\vec{n}).$$
It follows from Condition (2) of Definition \ref{def:free-evolution-equation} that
\begin{align*}
(1-q) \sum_{j=1}^{y_N} a_{N} \nabla_j u(t;\vec{n}) &= (1-q) \sum_{j=1}^{y_N} a_{N} q^{y_N - j} \nabla_{y_N} y(t;\vec{n})\\
&= (1-q) a_{N} \nabla_{y_N} y(t;\vec{n}) \sum_{j=1}^{y_N} q^{y_N - j}\\
&= a_N (1-q^{y_N}) \nabla_{y_N} u(t;\vec{n}).
\end{align*}
\item[(2)] Assume that $\vec{y} \in Y^N$ with $\sum_{i=1}^{N} y_i = k$ such that $y_0 > 0$. Then we would have $\vec{n}(\vec{y})$ satisfying $n_k = 0$. From Definition \ref{def:free-evolution-equation} Condition (3), it follows that $v(t;\vec{y}) = u(t;\vec{n}(\vec{y})) = 0$.
\item[(3)] For all $\vec{y} \in Y^N$, $v(0;\vec{y}) = u(0; \vec{n}(\vec{y})) = H(\vec{x},\vec{y})$.
\end{enumerate}
\end{proof}

\section{Nested Contour Integral Solution}
As discussed in Section \ref{evolution-equation-system}, existence of solutions to the \emph{free evolution equation system} is not clear in general. However, in this section, we are going to provide solutions to the \emph{free evolution equation system} with step and half stationary initial data in the form of nested contour integral.

\begin{theorem}
\label{thm:nested-contour-integration}
Fix $q \in (0,1)$, $a_i > 0$ for $i \ge 1$ and let $\vec{n} = (n_1,n_2, \dots, n_k).$ The \emph{free evolution equation system} as defined in Section \ref{evolution-equation-system} is solved by the following:
\begin{enumerate}
\item[(1)] For step initial data, 
\begin{align*}
\label{step-nested-contour}
u(t;\vec{n}) &= \frac{(-1)^k q^{k(k-1)/2}}{(2 \pi i)^k} \times \\
&\quad \int \dots \int \prod_{1 \le A < B \le k} \frac{z_A - z_B}{z_A - qz_B} \prod_{j=1}^k \left( \prod_{m=1}^{n_j} \frac{a_m}{a_m - z_j}\right) e^{(q-1)tz_j} \frac{dz_j}{z_j}, \numberthis
\end{align*}
where the integration contour for $z_A$ contains $\{qz_B\}_{B > A}$ and all $a_m's$ but not $0$.
\item[(2)] For half stationary initial data with parameter $\alpha > 0$, 
\begin{align*}
\label{half-stationary-nested-contour}
u(t;\vec{n}) &= \frac{(-1)^k q^{k(k-1)/2}}{(2 \pi i)^k} \times \\
& \quad \int \dots \int \prod_{1 \le A < B \le k} \frac{z_A - z_B}{z_A - qz_B} \prod_{j=1}^k \left( \prod_{m=1}^{n_j} \frac{a_m}{a_m - z_j}\right) e^{(q-1)tz_j} \frac{dz_j}{z_j - \alpha/q}, \numberthis
\end{align*}
where the integration contour for $z_A$ contains $\{qz_B\}_{B > A}$ and all $a_m's$ but not $\alpha / q$.
\end{enumerate}
\end{theorem}

\begin{corollary}
\label{nested-contour-corollary}
For q-TASEP with step initial data and $\vec{n} \in W^k_{>0}$, $$\mathbb{E} \left[ \prod_{j=1}^k q^{x_{n_j}+n_j} \right] = \frac{(-1)^k q^{k(k-1)/2}}{(2 \pi i)^k} \times \int \dots \int \prod_{1 \le A < B \le k} \frac{z_A - z_B}{z_A - qz_B} \prod_{j=1}^k \left( \prod_{m=1}^{n_j} \frac{a_m}{a_m - z_j}\right) e^{(q-1)tz_j} \frac{dz_j}{z_j},$$ where the integration contour for $z_A$ contains $\{qz_B\}_{B > A}$ and all $a_m's$ but not $0$.
\end{corollary}

\begin{proof}[Proof of Corollary \ref{nested-contour-corollary}]
Noting that the right hand side is simply a restriction of Equation (\ref{step-nested-contour}), and that $u(t;\vec{n})$ is equal to $\mathbb{E}^{\vec{x}}[H(\vec{x}(t), \vec{y})]$, it then suffices to show that $\mathbb{E}[H(\vec{x}(t), \vec{y})] =\mathbb{E} \left[ \prod_{i=1}^N q^{(x_i(t) + i) y_i} \right]= \mathbb{E} \left[ \prod_{j=1}^k q^{x_{n_j}+n_j} \right]$, where the expectation is taken with the step initial data. Recall that $y_i = |\{n_j: n_j = i\}|$, therefore
\begin{align*}
\prod_{i=1}^N q^{(x_i + i) y_i} = \prod_{i=1}^{N} \prod_{j=n_{i,1}}^{n_{i,1}+y_i-1} q^{x_{n_j} + n_j} = \prod_{j=1}^k q^{x_{n_j}+n_j},
\end{align*}
where $n_{i,1} = \min \{j : n_j = i\}$.
\end{proof}

\begin{proof}[Proof of Theorem \ref{thm:nested-contour-integration}]
We first prove the case for step initial data. That is, we need to show that the function $u(t;\vec{n})$ as defined satisfies the four conditions in Definition \ref{def:free-evolution-equation}, referred to as (1) - (4). 

\begin{enumerate}
\item[(1)] Let $$f(t;\vec{n}) = \prod_{j=1}^k \left( \prod_{m=1}^{n_j} \frac{a_m}{a_m - z_j}\right) e^{(q-1)tz_j}$$ such that $$u(t;\vec{n}) = \frac{(-1)^k q^{k(k-1)/2}}{(2 \pi i)^k} \times \int \dots \int \prod_{1 \le A < B \le k} \frac{z_A - z_B}{z_A - qz_B} f(t;\vec{n}) \prod_{j=1}^k \frac{dz_j}{z_j}.$$
Note that $$\frac{d}{dt} f(t;\vec{n}) = (q-1) \left( \sum_{i=1}^k z_i \right) f(t;\vec{n})$$ and that
\begin{align*}
\nabla_i f(t;\vec{n}) &= \prod_{j=1, j \neq i}^k \left( \prod_{m=1}^{n_j} \frac{a_m}{a_m - z_j}\right) e^{(q-1)tz_j} \prod_{m=1}^{n_i-1} \frac{a_m}{a_m - z_i} \left( 1 - \frac{a_{n_i}}{a_{n_i} - z_i} \right) \\
&= \prod_{j=1, j \neq i}^k \left( \prod_{m=1}^{n_j} \frac{a_m}{a_m - z_j}\right) e^{(q-1)tz_j} \prod_{m=1}^{n_i-1} \frac{a_m}{a_m - z_i} \left( -\frac{z_i}{a_{n_i} - z_i} \right)\\
&= - \frac{z_i}{a_{n_i}} f(t;\vec{n}) \numberthis \label{proof-nested-contour}
\end{align*}
Therefore, we have $$\frac{d}{dt} f(t;\vec{n}) = (1-q) \sum_{i=1}^k a_{n_i} \nabla_i f(t;\vec{n}).$$
Since $f(t;\vec{n})$ is the only component in $u(t;\vec{n})$ that is affected by the operators, by linearity, condition $(1)$ follows.
\item[(2)] Without loss of generality, assume $n_1 = n_2$. 

From Equation (\ref{proof-nested-contour}), we have that $\nabla_i u(t;\vec{n}) = - \frac{z_i}{a_{n_i}} u(t;\vec{n})$ and hence 
\begin{align*}
& \quad (\nabla_1 - q \nabla_2) u(t;\vec{n}) \\
&= - \frac{z_1 - qz_2}{a_{n_1}} u(t;\vec{n}) \\
&= \frac{(-1)^k q^{k(k-1)/2}}{(2 \pi i)^k} \times \int \dots \int \left( - \frac{z_1 - qz_2}{a_{n_1}} \right) \prod_{1 \le A < B \le k} \frac{z_A - z_B}{z_A - qz_B} f(t;\vec{n}) \prod_{j=1}^k \frac{dz_j}{z_j} \\
&= \frac{(-1)^k q^{k(k-1)/2}}{(2 \pi i)^k} \times \int \dots \int \left( - \frac{z_1 - z_2}{a_{n_1}} \right) \prod_{A<B, (A,B) \neq (1,2)} \frac{z_A - z_B}{z_A - qz_B} f(t;\vec{n}) \prod_{j=1}^k \frac{dz_j}{z_j} 
\end{align*}
From the last equality, the contours of $z_1$ and $z_2$ can be deformed to be the same without encountering any poles (since the term $z_1 - qz_2$ in the denominator has been canceled). Notice also that in the integrand the terms involving $z_1$ and the terms $z_2$ are separated into identical parts, and hence the integral can be re-written as 
$$(\nabla_1 - q \nabla_2) u(t;\vec{n}) = \int \int (z_1 - z_2) G(z_1) G(z_2) dz_1 dz_2,$$
where $G(z)$ incorporates the integration involving $z_3, z_4, \dots, z_k$. Since the contour of $z_1$ and $z_2$ are the same, we have that 
\begin{align*}
& \quad \int \int (z_1 - z_2) G(z_1) G(z_2) dz_1 dz_2 \\
&= \int z_1 G(z_1) dz_1 \times \int G(z_2) dz_2 - \int z_2 G(z_2) dz_2 \times \int G(z_1) dz_1 \\
&= 0.
\end{align*}
That is, $\nabla_1 u(t;\vec{n}) = q \nabla_2 u(t;\vec{n})$.

\item[(3)] Notice that poles corresponding to $z_k$ can only occur at the term $\prod_{m=1}^{n_k} \frac{a_m}{a_m - z_k}$. When $n_k = 0$, the term vanishes and hence the integration with respect to $z_k$ also vanishes by Cauchy's integral theorem. Therefore, $u(t;\vec{n}) = 0$ when $n_k = 0$ follows. 

\item[(4)] For step initial data, we have $$u(0,\vec{n}) = H(\vec{x}(0),\vec{y}(\vec{n})) = \prod_{i=0}^{N} q^{(x_i(0) + i)y_i} = 1$$ for all $\vec{n} \in W^k_{>0}$ since $x_i(0) = -i$. It then suffices to show that $u(0,\vec{n}) = 1$ for all $\vec{n} \in W^k_{>0}$. 

We show this by directly evaluating the integrals using the technique of contour deformation. Consider first the contour for $z_1$. It can be expanded to infinity and the only poles it can possibly encounter in this process are $z_1 = 0$ and $z_1 = \infty$. However, $z_1 = \infty$ is not a pole because as $z_1 \rightarrow \infty$, the integrand does not go $\infty$ because of the decay in $\frac{a_m}{a_m - z_1}$. Therefore, the only pole encountered is $z_1 = 0$ and the contribution is calculated to be $$-q^{-(k-1)} \frac{(-1)^k q^{k(k-1)/2}}{(2 \pi i)^{k-1}} \times \int \dots \int \prod_{2 \le A < B \le k} \frac{z_A - z_B}{z_A - qz_B} \prod_{j=2}^k \left( \prod_{m=1}^{n_j} \frac{a_m}{a_m - z_j}\right) \frac{dz_j}{z_j}.$$ Continuing this process for $z_2, \dots, z_{k}$ will bring down a factor of $-q^{-(k-2)}, \dots, -q^{0}$ respectively, from which we conclude that $u(0;\vec{n}) = 1$ as desired.
\end{enumerate}

Next, we show the case for half stationary initial data. Condition $(1) - (3)$ follows in exactly the same way as that for step initial data. Therefore, we only need to show that Condition $(4)$ follows. To show this, we need a lemma. 

\begin{lemma}
\label{q-geometric-expectation-lemma}
Fix $r \ge 1$. If $X$ is a q-Geometric random variable with parameter $\alpha \in [0,1)$, then $$\mathbb{E}\left[ q^{-rX} \right] = \prod_{i=1}^{r} \frac{1}{1- \alpha / q^i}$$ so long as $\alpha q^{-r} < 1$; and otherwise the expectation is infinite.
\end{lemma}

\begin{proof}[Proof of Lemma \ref{q-geometric-expectation-lemma}]
Recall the q-Binomial theorem $$\sum_{n=0}^{\infty} \frac{(a;q)_n}{(q;q)_n} z^n = \frac{(az;q)_{\infty}}{(z;q)_{\infty}}$$
Therefore, we have 
\begin{align*}
\mathbb{E} \left[ q^{-rX} \right] &= (\alpha; q)_{\infty} \sum_{k=0}^{\infty} \frac{(\alpha / q^r)^k}{(q;q)_k} = (\alpha; q)_{\infty} \sum_{k=0}^{\infty} \frac{(0;q)_k}{(q;q)_k} (\alpha / q^r)^k\\
																	&= \frac{(\alpha; q)_{\infty} }{(\alpha q^{-r}; q)_{\infty}} = \prod_{i=1}^{r} \frac{1}{1- \alpha / q^i}
\end{align*}
\end{proof}
For half-stationary initial data, we have that $$\prod_{i=1}^{k} q^{x_{n_i}(0) + n_i} = \prod_{i=1}^{k} q^{-\sum_{m=1}^{n_i} X_m} = \prod_{i=1}^{k} \prod_{m=n_{i+1} +1}^{n_i} q^{-iX_m},$$ where $X_i$'s are independent q-Geometric random variables. By taking expectation and using the independence, we have that $$\mathbb{E} [H(\vec{x}; \vec{y}(\vec{n})] =  \prod_{i=1}^{k} \prod_{m=n_{i+1} +1}^{n_i} \mathbb{E}\left[ q^{-iX_m} \right] = \prod_{i=1}^{k} \prod_{m=n_{i+1} +1}^{n_i} \prod_{j=1}^{i} \frac{a_m}{a_m - \alpha / q^j}.$$

It then suffices to show that $u(0;\vec{n}) = \mathbb{E} [H(\vec{x}; \vec{y}(\vec{n})]$. We prove this by directly evaluating the nested contour integrals at $t = 0$ using the same technique as in that for step initial data. Consider first the contour for $z_1$ and expand it to infinity. The only pole encountered is $z_1 = \alpha / q$. Evaluating the residue, we get that 
\begin{align*}
u(0;\vec{n}) &= \prod_{m=1}^{n_1} \frac{a_m}{a_m - \alpha / q} \times \frac{(-1)^{k-1} q^{(k-1)(k-2)/2}}{(2 \pi i)^{k-1}} \times \\
& \quad \int \dots \int \prod_{2 \le A < B \le k} \frac{z_A - z_B}{z_A - qz_B} \prod_{j=2}^k \left( \prod_{m=1}^{n_j} \frac{a_m}{a_m - z_j}\right) \frac{dz_j}{z_j - \alpha/q^2}
\end{align*}
Continuing this process for $z_2, \dots, z_k$, we get the equality as desired. 
\end{proof}